\documentclass[a4paper,12pt]{report}
\usepackage[french]{babel}
\usepackage[utf8]{inputenc}
\usepackage{graphicx}
\usepackage{hyperref}
\usepackage{geometry}
\geometry{a4paper, margin=1in}

\begin{document}

\begin{titlepage}
    \centering
    \includegraphics[width=0.5\textwidth]{logo_heh.png}\par\vspace{1cm}
    {\scshape\LARGE Haute École en Hainaut \par}
    \vspace{1cm}
    {\huge\bfseries Projet Interdisciplinaire : SmartCity \par}
    \vspace{2cm}
    {\Large\itshape Groupe 9 \par}
    \vspace{1cm}
    {\Large\itshape Noah Debay, De Coster Koryan, Di Letto Matteo, Florian Olivier, Augustin Vangeebergen \par}
    \vfill
    {\large Bachelier en Informatique - Orientation réseaux \& télécommunications - Bloc 2 \par}
    \vfill
    {\large Décembre 2024 \par}
\end{titlepage}

\tableofcontents
\newpage

\chapter{Introduction}
Ce rapport présente le projet interdisciplinaire réalisé dans le cadre du Bachelier en Informatique, orientation réseaux \& télécommunications, bloc 2. Le projet vise à simuler la gestion d'une ville intelligente en intégrant des aspects d'infrastructure réseau, de développement web, et de gestion collaborative.

\chapter{Description du Projet}
Le projet interdisciplinaire a pour objectif de nous immerger dans une simulation professionnelle qui allie infrastructure réseau, systèmes informatiques, bases de données, développement web, et gestion collaborative. Chaque groupe représente une entité indépendante (par exemple, une ville connectée dans notre cas) avec des responsabilités propres, tout en collaborant avec les autres pour contribuer à un système centralisé interconnecté.

\section{Comportement}
Durant cette semaine de projet, nous avons dû faire preuve de professionnalisme. Et donc trouver une méthodologie de travail de groupe adéquate, une cohésion, un échange professionnel entre nous et les professeurs servant d'expert ou ceux servant de client. Il est donc important d'employer un vocabulaire adéquat en fonction de notre interlocuteur.

\section{Groupe}
L'élaboration du projet se fera comme convenu au départ par groupe de 5 étudiants maximum. Chaque groupe devra comporter plusieurs profils tel que :
\begin{itemize}
    \item 1 Chef d'équipe (Florian Olivier) : Il sera la personne ressource pour le client et sera donc l'UNIQUE référent pour celui-ci. 
    \item 2 développeurs (Matteo Di Letto, Noah Debay) 
    \item 2 gestionnaires réseau (Augustin Vangeebergen,Koryan De Coster) 
\end{itemize}

\section{Évaluation}
L'évaluation sera réalisée à différents niveaux :
\begin{itemize}
    \item Présentation du produit final aux clients.
    \item Le code.
    \item Configuration réseau et infrastructure.
    \item La rédaction du rapport.
    \item La documentation demandée dans le cahier de charges.
\end{itemize}

\chapter{Objectifs Généraux}
Le projet vise à développer une simulation de gestion pour une ville intelligente. Chaque groupe représente un département crucial (transport, énergie, sécurité, eau) et est responsable de gérer les données locales tout en contribuant à une base centrale pour fournir des services partagés et un tableau de bord global (en Django).

\chapter{Infrastructure Réseau \& Systèmes}
\section{Configuration Réseau}
Chaque classe disposera d'un switch configuré avec les VLAN suivants :
\begin{itemize}
    \item VLAN 10, 20, 30, 40 : Attribués aux groupes 1 à 4.
    \item VLAN 100 : VLAN commun regroupant les services partagés communs.
\end{itemize}

\section{Serveurs}
Chaque groupe disposera de serveurs locaux et collaborera avec un serveur commun.
\begin{itemize}
    \item Serveur Windows Local (par VLAN)
    \item Serveur Windows commun (VLAN 100)
    \item Serveur Linux Local (par VLAN)
    \item Serveur Linux Commun (VLAN 100)
\end{itemize}

La configuration est en annexe.

\chapter{Développement des Applications}
\section{Applications Locales (PHP)}

Pour Matteo

\section{Applications Globales (Python ou PHP)}

L'application globale devait être impléméntée en Python, avec le framework Django. Malheureusement, à cause de notre retard et d'une mauvaise coordination, celui-ci n'a pas été réalisé. La maquette est cependant dsponible en annexe.

\section{Bonnes Pratiques}
Ces applications doivent répondre aux normes de bonnes pratiques vues en cours, être documentée, sécurisée, et ergonomique.

\chapter{Analyse}
Production des diagrammes UML :
\begin{itemize}
    \item Diagrammes de cas d'utilisation. Voir annexe
    \item Diagrammes de classes. Voir annexe
    \item Diagrammes de séquence. Non fait
    \item Diagramme MLD. Voir annexe
\end{itemize}

\chapter{Notions de Sécurité de l'AD et Journalisation de la Base de Données}
\section{Authentification}
Authentification des utilisateurs via l'Active Directory (local ou central) avant d'accéder aux applications.

\section{Contrôle d'Accès}
Permissions basées sur les groupes AD. Validation des actions critiques selon le rôle défini dans AD.

\section{Traçabilité}
A a méliorer : 
Journaux de connexion enregistrés sur le serveur AD pour chaque utilisateur. Suivi des modifications dans les bases de données via triggers.

\chapter{Documentation Technique}
Ddocumentation complète comprenant :
\begin{itemize}
    \item Plans réseau.
    \item Maquettes.
    \item Scripts de configuration.
    \item Rapports de tests.
\end{itemize}

\section{Explications du Linux local}

La machine Linux local est un fedora server 40. 
Cette distribution est pratique car elle inclut plus de packages qu'Alma Linux.
Une installation comprenant :

\begin{itemize}
\item httpd
\item PHP
\item PHPyadmin
\item File Browser
\item Partage Samba
\item MAriaDB
\end{itemize}

Niveau sécurité, on a :
\begin{itemize}
\item Mots de passe utilisateurs forts 
\item Firewall
\item SELinux
\item options de montage 
\end{itemize}

Backups : 

Sauvegarde journalière avec rsync : DB et site.

Sauvegarde avant les mises à jour (avec alias pour sauvegarder puis mettre à jour).

\section{Explications du Linux Global}

Idem que pour le linux local.

\section{Explications du code}

Matteo

\chapter{Planning Suggéré}
\section{Jour 1 : Analyse et Planification}
Étude des besoins spécifiques et communs. Création des diagrammes UML, modélisation de la base de données et des maquettes.

\section{Jour 2 : Configuration Réseau et Serveurs}
Configuration des VLAN. Installation et configuration des serveurs.

\section{Jour 3 : Développement Local (PHP)}
Implémentation des fonctionnalités CRUD locales. Tests unitaires sur les bases de données locales.

\section{Jour 4 : Application Commune (Python ou PHP)}
Développement des fonctionnalités globales. Tests d'intégration entre bases de données locales et centrale.

\section{Jour 5 : Validation et Présentation}
Validation réseau, serveurs, et applications. Présentation des résultats par chaque groupe.

\chapter{Responsabilités par Département}
\section{Transport (VLAN 10)}
Gestion des parkings intelligents, suivi du trafic et gestion des feux connectés, rapport des temps d'attente.

\section{Énergie (VLAN 20)}
Suivi de la consommation énergétique des panneaux solaires, gestion des alertes de surcharge, génération de rapports énergétiques.

\section{Sécurité (VLAN 30)}
Vidéosurveillance connectée, alerte anti-intrusion avec capteurs, suivi en temps réel des caméras.

\section{Eau (VLAN 40)}
Gestion des capteurs de qualité d'eau potable, détection des fuites d'eau, génération de rapports sur la consommation.

\chapter{Cahier des Charges par Département}
Chaque département est responsable de ses données locales et de leur synchronisation avec la base de données centrale. Les applications locales doivent inclure une interface simple, ergonomique et intuitive. Les permissions d'accès doivent être gérées via Active Directory et des validations côté serveur.

\chapter{Daily Scrum}

\section{Jour 1 : 16 décembre 2024}

\subsection{À Faire Aujourd'hui}
\begin{itemize}
    \item Étude des besoins spécifiques du département "Transport" et communs.
    \item Création des diagrammes UML de cas d'utilisation pour les fonctionnalités à développer (ex. : détection de place libre, rapport des temps d'attente).
    \item Modélisation de la base de données et des maquettes.
    \item Configurer le VLAN pour le département Transport.
    \item Établir les exigences pour l'infrastructure réseau (serveurs nécessaires, configuration IP).
    \item Documenter les besoins en matière de sécurité et d'accès pour le réseau.
\end{itemize}

\subsection{Fait Aujourd'hui}
\begin{itemize}
    \item Modélisation de la base de données globale et locale (Matteo)
    \item Découverte de mon rôle de chef d’équipe, évaluation des compétences, création de dossiers pour le suivi, début de l’UML, discussion sur la maquette du site web (Florian)
    \item Modélisation de la base de données locale, tests réseau, résolution de problèmes avec Noah, début de l’élaboration des layouts (Koryan)
    \item Création du repo git, branches pour le développement, plan de partitionnement des serveurs Linux, installation initiale des VM (Augustin)
    \item Compréhension des consignes, configuration du switch, configurations des VLANs et du SSH (Noah)
\end{itemize}

\subsection{À Faire Demain}
\begin{itemize}
    \item Commencer à développer l'application locale en PHP pour la gestion des parkings intelligents.
    \item Créer une interface utilisateur simple et ergonomique (HTML/CSS/JavaScript).
    \item Installer et configurer les serveurs nécessaires (DNS, DHCP, Web).
    \item Configurer les permissions d'accès via Active Directory pour le groupe Transport.
    \item Tester la connectivité réseau et s'assurer que le VLAN est opérationnel.
\end{itemize}

\subsection{Obstacles}

\begin{itemize}
    \item plusieurs erreur sur le switch Alcatel et test fait sur le routeur physique (changement de cable, reboot) (Noah)
    \item probleme global donc reunion pour mettre les maquettes en commun et ameliorer l'infra local
\end{itemize}

\subsection{Notes}
\begin{itemize}
    \item Cette journée a été consacrée à la modélisation de la base de données globale et locale.
    \item Florian a réalisé une évaluation des compétences pour choisir le rôle de chacun et a créé différents dossiers pour assurer un suivi.
    \item Koryan a travaillé sur la modélisation de la base de données locale et a débuté l’élaboration des layouts de l’application locale.
    \item Augustin a créé le repo git et a installé les VM.
    \item Noah a configuré le switch et les VLANs.
\end{itemize}

\section{Jour 2 : 17 décembre 2024}

\subsection{À Faire Aujourd'hui}
\begin{itemize}
    \item Commencer à développer l'application locale en PHP pour la gestion des parkings intelligents.
    \item Créer une interface utilisateur simple et ergonomique (HTML/CSS/JavaScript).
    \item Installer et configurer les serveurs nécessaires (DNS, DHCP, Web).
    \item Configurer les permissions d'accès via Active Directory pour le groupe Transport.
    \item Tester la connectivité réseau et s'assurer que le VLAN est opérationnel.
\end{itemize}

\subsection{Fait Aujourd'hui}
\begin{itemize}
    \item Échange avec les experts, mise en place et apprentissage de Django, activation de l'infrastructure Django, création du dépôt GitHub commun de la Smartcity avec Kilian (Florian)
    \item Fin de la conception des layouts (Koryan)
    \item Configuration des Vlans du switch. Debug de la connectivite (Ping VLAN vers VLAN). Configuration du site web HTML/CSS a partir d'images faites par mon coequipier. Construction de la DB sur phpMyAdmin. Commencement de quelques requetes PHP(Noah)
\end{itemize}

\subsection{À Faire Demain}
\begin{itemize}
    \item Implémenter les fonctionnalités CRUD (Créer, Lire, Mettre à jour, Supprimer) pour la gestion des parkings.
    \item Développer les fonctionnalités de suivi du trafic et de gestion des feux connectés.
    \item Effectuer des tests unitaires sur les fonctionnalités développées.
    \item Assurer le bon fonctionnement des serveurs et de la base de données.
    \item Tester la synchronisation des données entre l'application locale et la base de données centrale.
    \item 
\end{itemize}

\subsection{Obstacles}

\begin{itemize}
    \item Probleme sur le switch Alcatel car on avait oublier de changer la configuration de base vue qu'on avait evolue le reseaux global avec les nouvelles demande de Mr Pietrzak pour le nombre d'utilisateurs pour chaque domaine local donc changement d'ip et de masque qui n'etais plus correct dans les route du switch.
    \item Ce premier probleme nous a cause des soucis dans notre ping dans notre VLAN 10, donc d'installation du serveur Windows local avec le global.
\end{itemize}

\subsection{Notes}
\begin{itemize}
    \item Florian a échangé avec les experts et a mis en place Django.
    \item Koryan a terminé la conception des layouts.
    \item Noah a configuré les VLANs et a commencé la configuration du site web.
\end{itemize}

\section{Jour 3 : 18 décembre 2024}

\subsection{À Faire Aujourd'hui}
\begin{itemize}
    \item Implémenter les fonctionnalités CRUD (Créer, Lire, Mettre à jour, Supprimer) pour la gestion des parkings.
    \item Développer les fonctionnalités de suivi du trafic et de gestion des feux connectés.
    \item Effectuer des tests unitaires sur les fonctionnalités développées.
    \item Assurer le bon fonctionnement des serveurs et de la base de données.
    \item Tester la synchronisation des données entre l'application locale et la base de données centrale.
\end{itemize}

\subsection{Fait Aujourd'hui}
\begin{itemize}
    \item mise en place du serveux windows local configurer (Florian)
\end{itemize}

\subsection{À Faire Demain}
\begin{itemize}
    \item Intégrer les fonctionnalités développées avec l'application globale (partage des données avec d'autres départements).
    \item Développer des rapports sur les temps d'attente et la gestion des parkings.
    \item Vérifier l'intégration des systèmes et s'assurer que les données circulent correctement entre les applications locales et la base de données centrale.
    \item Mettre en place des mesures de sécurité pour protéger les données de transport.
\end{itemize}

\subsection{Obstacles}
Aucun obstacle signalé pour le moment.

\subsection{Notes}
\begin{itemize}
    \item
\end{itemize}

\section{Jour 4 : 19 décembre 2024}

\subsection{À Faire Aujourd'hui}
\begin{itemize}
    \item Intégrer les fonctionnalités développées avec l'application globale (partage des données avec d'autres départements).
    \item Développer des rapports sur les temps d'attente et la gestion des parkings.
    \item Vérifier l'intégration des systèmes et s'assurer que les données circulent correctement entre les applications locales et la base de données centrale.
    \item Mettre en place des mesures de sécurité pour protéger les données de transport.
\end{itemize}

\subsection{Fait Aujourd'hui}
\begin{itemize}
    \item
\end{itemize}

\subsection{À Faire Demain}
\begin{itemize}
    \item Finaliser l'application locale et effectuer des tests d'intégration.
    \item Préparer une présentation des résultats et des fonctionnalités développées.
    \item Valider la configuration réseau et s'assurer que tout fonctionne comme prévu.
    \item Préparer la documentation technique sur la configuration réseau et les serveurs.
\end{itemize}

\subsection{Obstacles}
Aucun obstacle signalé pour le moment.

\subsection{Notes}
\begin{itemize}
    \item
\end{itemize}

\section{Jour 5 : 20 décembre 2024}

\subsection{À Faire Aujourd'hui}
\begin{itemize}
    \item Finaliser l'application locale et effectuer des tests d'intégration.
    \item Préparer une présentation des résultats et des fonctionnalités développées.
    \item Valider la configuration réseau et s'assurer que tout fonctionne comme prévu.
    \item Préparer la documentation technique sur la configuration réseau et les serveurs.
\end{itemize}

\subsection{Fait Aujourd'hui}
\begin{itemize}
    \item
\end{itemize}

\subsection{À Faire Demain}
\begin{itemize}
    \item
\end{itemize}

\subsection{Obstacles}
Aucun obstacle signalé pour le moment.

\subsection{Notes}
\begin{itemize}
    \item
\end{itemize}

\end{document}
